\documentclass[12]{article}
\usepackage{rajeevresume}
\usepackage{./latex-emoji/emoji}
\begin{document}

\chead
{
  \huge Rajeev Atla
}

% \write18{inkscape -D markdown.svg -o markdown.pdf --export-latex}
% \write18{inkscape -D -h 32 -w 32 ubuntu.svg -o ubuntu.pdf --export-latex}


\reversemarginpar
\marginnote{
  \begin{flushleft}
    {\large \emoji{waving-hand} \textbf{Contact}} \\
    \href{https://rajeevatla.com}{\textcolor{blue}{\faGlobe}\ rajeevatla.com} \\
    \href{mailto:hi@rajeevatla.com}{\textcolor{green}{\faEnvelopeSquare}\ hi@rajeevatla.com} \\
    \href{https://github.com/RajeevAtla}{\textcolor{ghred}{\faGithub}\ RajeevAtla} \\
    \href{https://www.linkedin.com/in/rajeev-atla/}{\textcolor{linkedin}{\faLinkedinIn}\  rajeev-atla} \\
    \href{https://www.facebook.com/rajeevratla}{\textcolor{fb}{\faFacebook}\ rajeevratla} \\
    {\large \emoji{keyboard} \textbf{Programming}} \\
    \textcolor{python}{\faPython}\ Python \\
    \faJava\ Java \\
    \textcolor{Goldenrod}{\faJsSquare}\ JavaScript \\
    \textcolor{Turquoise}{\faDatabase}\ SQL \\
    {\large \emoji{laptop} \textbf{Technologies}} \\
    \input{numpy.pdf_tex} NumPy \\
    \input{pytorch.pdf_tex} PyTorch \\
    \input{pandas.pdf_tex} Pandas \\
    \input{scikitlearn.pdf_tex} Scikit-learn \\
    \faGit*\ Git \\
    {\large \emoji{writing-hand} \textbf{Markup}} \\
    \input{latex-icon.pdf_tex} \LaTeX \\
    \asyinclude[width = 28pt, height = 16pt, keepAspect = false]{logo.asy}\ Asymptote \\
    \textcolor{html5}{\faHtml5}\ HTML \\
    \textcolor{css3}{\faCss3}\ CSS \\
    \input{markdown.pdf_tex} Markdown \\
    {\large \emoji{hammer-and-wrench} \textbf{Tools}} \\
    \input{emacs.pdf_tex} Emacs \\
    \input{ubuntu.pdf_tex} Ubuntu \\
    \textcolor{windows}{\faWindows}\ Windows \\
  \end{flushleft}
}[-41.5pt]






\vspace{-1.5\baselineskip}


\section{\faGraduationCap\ Education}
\entry
{2021 -- 2025}
{Rutgers University --- New Brunswick}
{
  \begin{itemize}
  \item Triple major in Computer Engineering, Computer Science, and Statistics/Mathematics
  \item Minor in Data Science
  \item Potential Minor in Operations Research
  \item Potential Certificates in Quantitative Economics or Computational Economics
  \item Extracurriculars: Engineering Honors Academy, IEEE, Quantitative Finance Club, Competitive Programming Club, Engineering Honors Council, Alliance of Computer Scientists, Math Association, Statistics Club, Quidditch
  \end{itemize}
}

\entry
{2017 -- 2021}
{John P. Stevens High School}
{
  \begin{itemize}
  \item Scored a \textbf{35/36} on ACT
  \item Took \textbf{19} AP exams (National AP Scholar) and \textbf{8} honors classes
  \item \textbf{5.56} Weighted GPA \& \textbf{4.07} Unweighted GPA
  \item Extracurriculars: Science Bowl (President), Science League (President), Science Olympiad (President), Physics Club (President), Chemistry Club (President), Quiz Bowl, National Honor Society, Science National Honors Society, National Technical Honor Society, National English Honor Society, Mu Alpha Theta
  \end{itemize}
}

\entry
{2020 -- 2021}
{Columbia Science Honors Program}
{
  \begin{itemize}
  \item Took \emph{Introduction to Algorithms} and \emph{Graph Theory}
  \item Selected as one of \textbf{\textasciitilde 2000} applicants from the tri-state area (NJ, NY, CT)
  \end{itemize}
}



\section{\faSitemap\ Projects}

\entry
{2020 -- 2021}
{SuperconGAN: Superconductivity and GANs}
{
  \begin{itemize}
  \item Used PyTorch to construct and train a generative adversarial network (GAN) to analyze superconductivity data
  \item Withdrew data from UCI Machine Learning Repository using Pandas
  \item Published package on PyPI with \textbf{20,000+ downloads}
  \item Wrote unit tests using Pytest
  \item GitHub Repository: \url{https://github.com/RajeevAtla/SuperconGAN}
  \end{itemize}
}

\entry
{2019 -- 2021}
{\url{rajeevatla.com}}
{
  \begin{itemize}
  \item Used Jekyll and GitHub Pages to publish personal website
  \item Blogged on various technical subjects including physics and math
  \end{itemize}
}

\entry
{Aug. 2020}
{Sentiment Classification on IMDb Movie Reviews}
{
  \begin{itemize}
  \item Lead a team of \textbf{5} in using Sklearn and Pandas to construct an F1-based model to classify movie reviews as positive or negative
  \item Achieved \textbf{90.5\%} accuracy using linear bigram tf-idf
  \end{itemize}
}

\entry
{May 2021}
{Travel App}
{
  \begin{itemize}
  \item Designed mobile app to give iconic tours of areas along with 4 team members
  \item Wrote controllers and models for MongoDB using Mongoose ORM
  \item Utilized Flutter for frontend and Express.js and MongoDB for backend
  \item Placed \textbf{2nd} at HackExeter 2021
  \end{itemize}
}


\section{\faAward\ Awards}

\entry
{March 2020}
{US Physics Olympiad Qualifier}
{
  \begin{itemize}
  \item Placed in \textbf{top 400} out of \textbf{5,000+} on F=ma exam, based on knowledge of calculus-based mechanics and physical intuition
  \item Final exam cancelled due to COVID-19
  \item \url{https://www.aapt.org/physicsteam/2020/upload/2020-USAPhO-Qualifiers\_v3.pdf}
  \end{itemize}
}

\entry
{2019, 2021}
{NJIT Chemistry Olympics}
{
  \begin{itemize}
  \item Utilized knowledge of organic and inorganic chemical nomenclature, as well as general chemistry knowledge
  \item Lead nomenclature team to \textbf{3rd place in 2021} and \textbf{4th place in 2019} (2020 canceled due to COVID-19)
  \item Lead demonstration show team to \textbf{4th place in 2019}
  \item Selected to represent high school out of \textbf{200+} applicants
  \end{itemize}
}

\end{document}
